\documentclass[12pt]{article}
\usepackage[margin=0.5in]{geometry}

\begin{document}
    CSC311 Reflection \hfill Guanglei Zhu \smallskip
    \hrule \bigskip
    \textbf{Suggestion used: giving users greater control over their feeds}. \bigskip \par
    There are several ways to give users more control over their feeds. One possible approach is to allow users 
    turning on or off personalized recommendation completely. This includes generating the feeds based on user's browsing history, 
    geographic location or time spent viewing etc. Consequently, it could potentially mitigate the users from being 
    manipulated by the feed. Nowadays, this policy has been implemented widely. For example, in the recent IOS update, 
    users are given an option to choose allow personalized recommendation or not for all the applications. By giving users full control over 
    the usage of personalized information, it should influence its users in a more ethical way. In addition, it might also 
    improve data privacy which is another notorious issue.\par
    The recommender system could also provide a list of topics that the user could choose to like or dislike, then the system would
    adjust the feed accordingly. Instead of letting the recommender system choosing which content to amplify algorithmically, users should
    also be given option to reduce the feed of a particular subject. By being able to adjust the feed by themselves, users are less
    likely to be influenced by others. Reducing the algorithmic behavior and suggesting user intervention to the recommender system 
    in theory ease the manipulation. Moreover, this improvement is also conducted in a ethical way, since the users get to choose 
    the content feed by themselves. \par 
    In general, I think giving users greater control over their feeds could be a viable way to reduce the influence from others. At the
    same time, this approach also remains being ethical to the users.
\end{document}